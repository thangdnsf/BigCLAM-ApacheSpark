 
\begin{abstracts}  
\nomenclature[KN]{BigCLAM}{Cluster Affiliation Model for Big Networks}   
Với sự phát triển nhanh chóng của các cộng đồng mạng kết nối như mạng xã hội Facebook, Twitter, Youtube,\dots hay mạng thương mại điện tử như Amazon, Netfix, Lazada,\dots việc phân tích các mạng này đóng vai trò ngày càng quan trọng, thời sự và được sự quan tâm nghiên cứu thuộc nhiều lĩnh vực như xã hội học, kinh tế, khoa học máy tính,\dots Trong đó bài toán phát hiện cộng đồng trong các mạng tương tác đang là một trong những nội dung nghiên cứu dành được nhiều sự quan tâm từ phía các nhà khoa học. Các nhà khoa học đã đề xuất rất nhiều các phương pháp phát hiện cộng đồng trong mạng tương tác, đặc biệt là các mạng có tính chồng chéo, trong số đó mô hình BigCLAM (Cluster Affiliation Model for Big Networks) được Yang.J và Leskovec.L đề xuất năm 2013 được chứng mình là khá nổi trội trong việc phát hiện cộng đồng chồng chéo trong mạng có kích thước lớn.
\nomenclature[KN]{SNAP}{Stanford Network Analysis Project} 

Đề tài này sẽ đi sâu vào nghiên cứu mô hình BigCLAM và đề xuất một vài phương pháp huấn luyện cũng như phương pháp tối ưu tốc độ huấn luyện. Trong quá trình thực nghiệm khóa luận, phương pháp được cài đặt trên hệ thống tính toán phân tán và sử dụng mạng có kích thước lớn trên trang SNAP\footnote{http://snap.stanford.edu} của Đại học Stanford như Facebook, Amazon, Youtube, \dots

Nội dung của khóa luận gồm 5 chương:

\textbf{Chương 1: Tổng quan về mạng và bài toán phát hiện cộng đồng chồng chéo}. Tại chương này của khóa luận trình một cách khái quát về mô hình mạng tương tác trong thực tế và các bài toán liên quan đặc biệt là bài toán phát hiện cộng đồng. Từ đó thiết lập động lực và mục tiêu của khóa luận này.

\textbf{Chương 2: Cơ sở lý thuyết}. Trong chương này sẽ trình bày chi tiết các khái niệm cơ sở được sử dụng trong khóa luận.

\textbf{Chương 3: Phương pháp phát hiện cộng đồng sử dụng mô hình BigCLAM}. Chương này sẽ trình bày một cách chi tiết về mô hình BigCLAM sử dụng phương pháp cực đại hàm khả dĩ để giải quyết bài toán phát hiện cộng đồng.

\textbf{Chương 4: Cài đặt phương pháp phát hiện cộng đồng sử dụng mô hình BigCLAM trên hệ thống phân tán}. Tại chương này sẽ giới thiệu mô hình tính toán phân tán Apache Hadoop và Apache Spark trong xử lý dữ liệu lớn. Đồng thời đề xuất mô hình cài đặt BigCLAM mới chạy trên hệ thống tính toán phân tán.

\textbf{Chương 5: Kết luận và hướng phát triển}. Đưa ra những kết quả của khóa luận và trình bày phương hướng phát triển sau khóa luận.

\end{abstracts}
 